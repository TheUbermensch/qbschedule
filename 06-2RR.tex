%%%%%%%%%%%%%%%%%%%%%%%%%%%%%%%%%%%%%%%%%%%%%%%%%%%%%%%%%%%%%%%%%%%%%%%%%%%%%%%
%%%%%%%%%%%%%%%%%%%%%%%%%%%%%%%%%%%%%%%%%%%%%%%%%%%%%%%%%%%%%%%%%%%%%%%%%%%%%%%
%%%%%                                                                     %%%%%
%%%%%    Template for a six-team double round robin quizbowl schedule.    %%%%%
%%%%%    Copyright 2014, Jonah Greenthal <jonah@jonahgreenthal.com>       %%%%%
%%%%%    You can use, modify, and distribute this in any way you want,    %%%%%
%%%%%    as long as you give credit.                                      %%%%%
%%%%%                                                                     %%%%%
%%%%%%%%%%%%%%%%%%%%%%%%%%%%%%%%%%%%%%%%%%%%%%%%%%%%%%%%%%%%%%%%%%%%%%%%%%%%%%%
%%%%%%%%%%%%%%%%%%%%%%%%%%%%%%%%%%%%%%%%%%%%%%%%%%%%%%%%%%%%%%%%%%%%%%%%%%%%%%%

\documentclass[landscape]{qbschedule}
\usepackage[top=0.5in, bottom=0.5in, left=0.5in, right=0.5in]{geometry}

\def\TournamentName{Six-Team Double Round Robin}
\def\TournamentSubtitle{\LaTeX\ Schedule Template}
\def\Phase{Preliminary Rounds (1--10)}
\def\ControlRoom{Room Z}
\def\MainFont{Computer Modern}

\newarray\Round
\Round(1)={Round 1}
\Round(2)={Round 2}
\Round(3)={Round 3}
\Round(4)={Round 4}
\Round(5)={Round 5}
\Round(6)={Round 6}
\Round(7)={Round 7}
\Round(8)={Round 8}
\Round(9)={Round 9}
\Round(10)={Round 10}

\newarray\Seed
\Seed(1)={Team Seeded First}
\Seed(2)={Team Seeded Second}
\Seed(3)={Team Seeded Third}
\Seed(4)={Team Seeded Fourth}
\Seed(5)={Team Seeded Fifth}
\Seed(6)={Team Seeded Sixth}

\newarray\Room
\Room(1)={Room A} % Average difference between seeds: 2.4
\Room(2)={Room B} % Average difference between seeds: 2.1
\Room(3)={Room C} % Average difference between seeds: 2.5

\setbool{ShowReaders}{true}
\newarray\Reader
\Reader(1)={Al Anderson}
\Reader(2)={Betsy Baxter}
\Reader(3)={\switch{Charlie Castro}}

\setbool{ShowScorekeepers}{true}
\newarray\Scorekeeper
\Scorekeeper(1)={Ann Allen}
\Scorekeeper(2)={Bill Berger}
\Scorekeeper(3)={\switch{Carly Carter}}

\setbool{ShowBuzzerAssignments}{true}
\newarray\Buzzer
\Buzzer(1)={\Seed(3)}
\Buzzer(2)={\Seed(1)}
\Buzzer(3)={\Seed(5)}

% It is generally not necessary to edit anything below this line, but hey, do whatever you want.
\setlength{\StaffSize}{14pt}
\setlength{\BuzzerSize}{14pt}
\setlength{\TeamNameSize}{11pt}
\setlength{\RoomColumnWidth}{2.755in}

\begin{document}
\begin{center}
	\rowcolors*{1}{white}{gray}
	\begin{tabular}{|y{\LeftColumnWidth}x{\RoomColumnWidth}x{\RoomColumnWidth}x{\RoomColumnWidth}|}
		\hline
		\rowcolor{black}
		\room{Room}				&	\room{\Room(1)}					&	\room{\Room(2)}					&	\room{\Room(3)}					\tabularnewline
		\ifbool{ShowReaders}{
		\staff{Reader}			&	\staff{\Reader(1)}				&	\staff{\Reader(2)}				&	\staff{\Reader(3)}		\tabularnewline
		}{}
		\ifbool{ShowScorekeepers}{
		\staff{Scorekeeper}		&	\staff{\Scorekeeper(1)}			&	\staff{\Scorekeeper(2)}			&	\staff{\Scorekeeper(3)}	\tabularnewline
		}{}
		\ifbool{ShowBuzzerAssignments}{
		\buzzer{Buzzer System}	&	\buzzer{\Buzzer(1)}				&	\buzzer{\Buzzer(2)}				&	\buzzer{\Buzzer(3)}				\tabularnewline
		}{}
		\hline
		\round{\Round(1)}				&	\matchup{\Seed(5)}{\Seed(3)}	&	\matchup{\Seed(1)}{\Seed(4)}	&	\matchup{\Seed(6)}{\Seed(2)}	\tabularnewline
		\round{\Round(2)}				&	\matchup{\Seed(2)}{\Seed(5)}	&	\matchup{\Seed(6)}{\Seed(1)}	&	\matchup{\Seed(4)}{\Seed(3)}	\tabularnewline
		\round{\Round(3)}				&	\matchup{\Seed(3)}{\Seed(1)}	&	\matchup{\Seed(5)}{\Seed(6)}	&	\matchup{\Seed(2)}{\Seed(4)}	\tabularnewline
		\round{\Round(4)}				&	\matchup{\Seed(6)}{\Seed(4)}	&	\matchup{\Seed(2)}{\Seed(3)}	&	\matchup{\Seed(1)}{\Seed(5)}	\tabularnewline
		\round{\Round(5)}				&	\matchup{\Seed(1)}{\Seed(2)}	&	\matchup{\Seed(4)}{\Seed(5)}	&	\matchup{\Seed(3)}{\Seed(6)}	\tabularnewline
		\hline
		\round{\Round(6)}				&	\matchup{\Seed(6)}{\Seed(2)}	&	\matchup{\Seed(5)}{\Seed(3)}	&	\matchup{\Seed(1)}{\Seed(4)}	\tabularnewline
		\round{\Round(7)}				&	\matchup{\Seed(4)}{\Seed(3)}	&	\matchup{\Seed(2)}{\Seed(5)}	&	\matchup{\Seed(6)}{\Seed(1)}	\tabularnewline
		\round{\Round(8)}				&	\matchup{\Seed(2)}{\Seed(4)}	&	\matchup{\Seed(3)}{\Seed(1)}	&	\matchup{\Seed(5)}{\Seed(6)}	\tabularnewline
		\round{\Round(9)}				&	\matchup{\Seed(1)}{\Seed(5)}	&	\matchup{\Seed(6)}{\Seed(4)}	&	\matchup{\Seed(2)}{\Seed(3)}	\tabularnewline
		\round{\Round(10)}				&	\matchup{\Seed(3)}{\Seed(6)}	&	\matchup{\Seed(1)}{\Seed(2)}	&	\matchup{\Seed(4)}{\Seed(5)}	\tabularnewline
		\hline
	\end{tabular}
\end{center}

\end{document}
